\documentclass{article}
\usepackage[utf8]{inputenc}
\usepackage[a4paper, top=0.5cm, bottom=2cm, left=2cm, right=2cm]{geometry}
\usepackage[portuguese]{babel}


\title{{\normalsize Faculdade de Engenharia da Universidade do Porto}\\Sistemas Embarcados --- Proposta de Projeto\\\textbf{Advanced $\mu$kernel on Arduino}}
\author{André Campanhã\\\texttt{up201806518@fe.up.pt} \and Miguel Almeida\\\texttt{up201806205@fe.up.pt} \and Rúben Almeida\\\texttt{up201704618@fe.up.pt}}

\date{Grupo A\\\vspace{0.5em}\today}

\begin{document}

\maketitle

\section{Hardware Necessário}

\begin{itemize}
    \item 3$\times$ Arduino Uno
    \item 3$\times$ Multi-function Shield
    \item 3$\times$ Cabo USB-B $\longleftrightarrow$ USB-A
\end{itemize}

\section{Descrição do Projeto}

\subsection{Objetivos do Projeto}
% Develop a tick based ukernel for the Arduino, supporting fixed priority multi-threading, full pre-emption, with independent stacks for each thread. (requires some assembly programming skills. Assembly can be embedded in C code)

O objetivo principal do projeto é desenvolver um \textit{micro-kernel} baseado em \textit{ticks} para o \textit{Arduino Uno}, que suporte \textit{multi-threading} e \textit{full pre-emption} com \textit{stacks} independentes para cada \textit{thread}. Para além disso, o grupo pretende implementar \textit{mutexes} com \textit{priority inheritance}. Desta forma o kernel desenvolvido passará a conseguir ter tarefas que utilizem recursos partilhados.

Como tarefa secundária, caso se verifique temporalmente exequível, irá ser implementado funcionalidades de Hierarchical Scheduling, de forma a ser possível correr tarefas com diferentes políticas de escalonamento no mesmo CPU, à semelhança do que acontece num sistema Linux.

\subsection{Arquitetura de Alto Nível do Software}

O grupo pretende comprovar a escalonabilidade de um sistema composto pelas seguintes cinco tarefas concorrentes:

\subsubsection{Task 1 --- Contador de Segundos}

Uma tarefa de baixa prioridade pretende manipular o mostrador de 7 segmentos do Arduino de forma a apresentar um contador de segundos. Desta forma, teremos uma task com periodicidade de um segundo a desempenhar esta funcionalidade.

\subsubsection{Task 2 --- Ativação dos LEDs}
\label{task_2}
Uma tarefa de baixa prioridade pretende ativar os LEDs do Arduino Uno de forma periódica. O número de LEDs a serem acesos em cada ativação dependerá de uma variável partilhada \texttt{number\_leds}.

\subsubsection{Task 3 --- Manipulação do Número de LEDs}

Uma tarefa de prioridade intermédia irá manipular a variável \texttt{number\_leds}. Em resultado da ativação desta tarefa o número de LEDs acesos pelo Arduino Uno irá variar ao longo da execução. A variável \texttt{number\_leds} irá ser partilhada com a Task 2 (\ref{task_2})

\subsubsection{Task 4 --- Buzzer ativo por botão}

Uma tarefa de baixa prioridade irá ativar o buzzer do Arduino em reação ao click num dos botões do hardware. Esta tarefa será assim de caraterísticas aperiódicas, e dependerá, somente, da atuação do utilizador humano.

\subsubsection{Task 5 --- Tarefa Full Stop}

Uma tarefa de prioridade alta irá ativar-se durante o período que um dos botões estiver a ser premido. Esta tarefa de perfil aperiódico tem como objetivo desestabilizar a escalonabilidade do sistema. Recorrendo à sua elevada prioridade pretende-se monopolizar os recursos do CPU. Quando ativa, esta tarefa irá paralisar todas as restantes que apresentem uma menor prioridade. 

\end{document}
