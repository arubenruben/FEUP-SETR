\documentclass{article}
\usepackage[utf8]{inputenc}
\usepackage[a4paper, top=0.5cm, bottom=2cm, left=2cm, right=2cm]{geometry}
\usepackage[portuguese]{babel}


\title{{\normalsize Faculdade de Engenharia da Universidade do Porto}\\Sistemas Embarcados --- Proposta de Projeto\\\textbf{Advanced $\mu$kernel on Arduino}}
\author{André Campanhã\\\texttt{up201806518@fe.up.pt} \and Miguel Almeida\\\texttt{up201806205@fe.up.pt} \and Rúben Almeida\\\texttt{up201704618@fe.up.pt}}

\date{Grupo A\\\vspace{0.5em}\today}

\begin{document}

\maketitle

\section{Hardware Necessário}

\begin{itemize}
    \item 3$\times$ Arduino Uno
    \item 3$\times$ Multi-function Shield
    \item 3$\times$ Cabo USB-B $\longleftrightarrow$ USB-A
\end{itemize}

\section{Descrição do Projeto}

\subsection{Objetivos do Projeto}
% Develop a tick based ukernel for the Arduino, supporting fixed priority multi-threading, full pre-emption, with independent stacks for each thread. (requires some assembly programming skills. Assembly can be embedded in C code)

O objetivo principal do projeto é desenvolver um \textit{micro-kernel} baseado em \textit{ticks} para o \textit{Arduino Uno}, que suporte \textit{multi-threading} e \textit{full pre-emption} com \textit{stacks} independentes para cada \textit{thread}. Além disso, implementaríamos também \textit{mutexes} com \textit{priority inheritance}, de forma a conseguirmos ter tarefas que utilizem recursos partilhados.

Como tarefa secundária, 

\subsection{Arquitetura de Alto Nível do Software}

O grupo pretende comprovar a escalonibilidade de um sistema composto pelas seguintes cinco tarefas:

\subsubsection{Task 1 --- Contador de Segundos}

Uma tarefa de baixa prioridade pretende manipular o mostrador de 7 segmentos de forma a apresentar um contador de segundos. Desta forma, teremos uma task de perfil periódico de segundo a segundo a desempenhar esta funcionalidade.

\subsubsection{Task 2 --- Ativação dos LEDs}
\label{task_2}
Uma tarefa de baixa prioridade pretende ativar os LEDs do Arduino Uno de forma periódica. O número de LEDs a serem acesos em cada ativação dependerá de uma variável \texttt{number\_leds}.

\subsubsection{Task 3 --- Manipulação do número dos LEDs}

Uma tarefa de prioridade intermédia irá manipular a variável partilhada \texttt{number\_leds}. A alteração deste valor resulta que a Task 2 irá acender diferentes números de LEDs ao longo do tempo.

\subsubsection{Task 4 --- Buzzer ativado por botão}

Uma tarefa irá ativar o buzzer do Arduino reagindo ao click de um dos botões.

\subsubsection{Task 5 --- Tarefa Full Stop}

Uma tarefa de prioridade alta irá ativar-se durante o período que um dos botões estiver a ser premido. Esta tarefa tem como objetivo desestabilizar a escalonabilidade do sistema, dado que pretende monopolizar os recursos do Scheduler, paralisando todas as restantes tasks menos prioritárias. 

\end{document}
